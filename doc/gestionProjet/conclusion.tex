Durant ces quatre mois, nous avons beaucoup appris sur SCRUM et l'Agile, grâce au projet, mais aussi grâce à Sandrine Maton, professionnelle et intervenante à l'UM2.

Le programme a été respecté, avec des sprints réguliers de deux semaines, tout en faisant évoluer nos objectifs.
Au cours de cette évolution, nous avons senti une réelle cohésion de groupe se former, notamment à partir du 3ème sprint où le fait de travailler en agile permis de mieux s'organiser et nous incita à communiquer fortement.

L'ajout de tâche au backlog n'ont pas perturbé notre rythme.
Nous avons donc fini notre projet dans les temps, avec quelques fonctionnalités supplémentaires qui n'étaient pas prévues.

