%comparatif avec le backInit
On peut voir sur le backlog \ref{bsp5} l'avancé du projet au sprint 5.
Nous avons rajouté des tâches, et des user-stories au fil du projet. En effet, nous avons eu de nouvelles idées à ajouter, et des fonctionnalités à écrire pour ne pas les oublier.
On peut aussi voir la justesse de nos estimations avec le temps réel passé sur chaque tâches.
Nous étions globalement dans les temps à chaque sprint, très peu de tâches ont été fini en retard.

{\tiny
\begin{longtable}[c]{|c|p{1.3cm}|c|p{3.5cm}|*{3}{c|}p{0.7cm}|}
\hline
\bf id & \bf Scénario utilisateur & \bf Priorité & \bf Tests & \bf Estimation & \bf Sprint & \bf Statut & \bf Temps réel \\
\hline
\endfirsthead
\hline
\bf id & \bf Scénario utilisateur & \bf Priorité & \bf Tests & \bf Estimation & \bf Sprint & \bf Statut & \bf Temps réel \\
\hline
\endhead
\hline
\caption{Backlog Sprint 5 (fin).} \label{bsp5}\\
\endlastfoot
\hline
\caption[]{Backlog Sprint 5.}\\
\endfoot
1 & L'utilisateur écrit du code dans un éditeur & 200 & L'utilisateur écrit du code dans un éditeur intégré & 20h & 5 &  &  \\
\hline
2 & L'utilisateur importe du code dans le logiciel & 1100 & Importer du code Stibbons depuis un fichier, vérifier que le code obtenu est bien identique à celui du fichier. & 4 & 1 & Fini & 1 \\
\hline
3 & L'utilisateur visualise les rapports d'erreurs du code & 400 & Exécuter pd 50 et constater une erreur. & 8h & 4 & Fini & 8h \\
\hline
4 & L'utilisateur visualise l'évolution du modèle & 1400 & Vérifier que l'interprétation d'instructions données fait bien évoluer comme prévu la tortue dans son environnement. & 24h & 1 & Fini & 70h \\
\hline
5 & L'utilisateur modifie la vitesse (pause, pas à pas, parallèle) & 700 & Faire varier la vitesse et observer le changement dans l'execution des tortues. & 12h & 4 & Fini & 14h \\
\hline
6 & Le «dieu-tortue» interprète le code de l'utilisateur & 1500 & Lancer l'interprétation pour~: repeat 4 { fd 1 rt 90 } (suivant syntaxe) ainsi que pour du code avec des erreurs~: repat 4 {...} par exemple, ou repeat 4 { fd 1 rt } & 12h & 1 & Fini & 32h \\
\hline
7 & L'utilisateur crée une nouvelle tortue (avec un code) & 600 & Lancer l'interprétation pour~: create-turtle {} et observer une nouvelle tortue apparaître dans l'interface graphique. & 7,5h & 2 & Fini & 4h \\
\hline
8 & Les tortues s'exécutent en parallèles & 1200 & Lancer l'interprétation pour deux tortues d'un bout de code et observer l'exécution parallèle via des écritures dans le terminal (Je suis tortue 1 et Je suis tortue 2 par exemple) & 42,5h & 2 & Fini & 40h \\
\hline
9 & Les tortues communiquent directement entre elles & 900 & Ecrire send(t,''Je suis là'') avec t une tortue qui a pour code~: m = recv() if (m == ''Je suis là'') fd 50 et observer la tortue avancer & 30h & 3 & Fini & 16h  \\
\hline
10 & Une tortue se déplace dans l'environnement & 1300 & Ecriture d'instructions simples~: repeat 4 { fd 1 rt 90 } (suivant syntaxe) - Renvoi de la position de la tortue après chaque deplacement~: where\_am\_i(); (suivant syntaxe) & 16h & 1 & Fini & 24h \\
\hline
11 & Les tortues communiquent avec les zones de l'environnement & 1000 & Ecrire if(zone.color == red) { color = blue } sur une zone de couleur rouge et observer la tortue changer de couleur. & 30h & 3 & Fini & 12h \\
\hline
12 & L'utilisateur exporte le code & 500 & L'utilisateur sauvegarde son code dans un fichier externe & 1h & 5 &  &  \\
\hline
13 & L'utilisateur exporte le modèle & 300 & Sauvegarder le modèle et observer les données en sortie (JSON) & 30h & 4 & Fini & 22h \\
\hline
14 & L'utilisateur ajoute une entrée & 100 &  &  &  &  &  \\
\hline
15 & L'utilisateur remet à zéro l'environnement & 800 & Remettre à zéro après une exécution et observer que le monde est vierge. & 20h & 4 & Fini & 23h \\
\hline
16 & L'utilisateur utilise des variables dans le code & 1275 & Ecrire a = 90 fd a et observer la tortue qui avance. & 12h & 2 & Fini & 3h \\
\hline
17 & L'utilisateur définit des fonctions personnalisées dans le code & 1250 & Ecrire function f () { fd 90 } f () et observer la tortue qui avance. & 23,5h & 2 & Fini & 11h \\
\hline
18 & Les tortues communiquent via l'environnement & 950 & Ecrire broadcast(20,''Je suis là'') et dans une autre tortue dans le rayon avec le code m = recv() if (m == ''Je suis là'') fd 50 et observer la tortue avancer. & 10h & 3 & Fini & 2h \\
\hline
19 & L'utilisateur utilise des conditionnelles & 550 & Ecrire if(false) { fd 90 } et observer que la tortue ne bouge pas. Refaire le même test avec if(true) et observer que la tortue bouge. & 3,5h & 2 & Fini & 3,5h \\
\hline
20 & L'utilisateur utilise des boucles & 575 & Ecrire pd repeat 4 { fd 40 rt 90 } et observer que la tortue dessine un carré. & 7h & 2 & Fini & 3,5h \\
\hline
21 & L'utilisateur définit des fonctions avec paramètres & 1225 & Ecrire function f(a) { fd a } f(90) et observer la tortue avancer. & 10h & 3 & Fini & 16h \\
\hline
22 & L'utilisateur instancie des agents avec paramètres & 560 & Ecrire agent wolf (a) { fd a } new wolf (50) et observer la nouvelle tortue avancer. & 10h & 3 & Fini & 16h \\
\hline
23 & L'utilisateur modifie la couleur d'un agent (tortue ou zone) & 450 & Ecrire new agent { color = red } et observer la nouvelle tortue rouge. & 3h & 3 & Fini & 2h \\
\hline
24 & La tortue accède aux données de son parent & 50 & Taper parent.color= blue et observer le parent devenir bleu & 1h & 5 & Fini & 1h \\
\hline
25 & L'utilisateur peut stocker des valeurs dans une table & 540 & On stocke le retour dans t= {1,2} & 10h & 4 & Fini & 9h \\
\hline
26 & L'utilisateur parcours un tableau & 530 & Faire foreach(f~: {1,2}) println(f) & 4h & 5 &  &  \\
\hline
27 & L'utilisateur lance l'application sans interface graphique & 250 & Lancer l'application et regarder les fichiers d'exportations & 16h & 5 & &  \\
\hline
\end{longtable}}
