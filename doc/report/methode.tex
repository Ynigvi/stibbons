Pour réaliser ce projet, nous avons choisi de faire de l'agile, avec la méthode SCRUM. Nous nous sommes principalement servi des cours de gestion de projet de ce semestre, en particulier des interventions de Sandrine Maton. Faire de l'Agile permet d'avoir un rendu fonctionnel à chaque fin de sprint, et donc de s'assurer d'avoir un rendu final. De plus, c'est une manière de faire qui se développe en entreprise, nous voulions donc l'expérimenter.
\subsubsection{Backlog}
Nous avons commencer le projet avec un sprint 0, au cours duquel nous avons défini l'univers du projet, et ses fonctionnalités (cf.~\ref{backlogv1}).

{\tiny
\begin{longtable}[c]{|c|p{1.3cm}|c|p{3.5cm}|*{3}{c|}p{0.7cm}|}
\hline
\bf id & \bf Scénario utilisateur & \bf Priorité & \bf Tests & \bf Estimation & \bf Sprint & \bf Statut & \bf Temps réel \\
\hline
\endfirsthead
\hline
\bf id & \bf Scénario utilisateur & \bf Priorité & \bf Tests & \bf Estimation & \bf Sprint & \bf Statut & \bf Temps réel \\
\hline
\endhead
\hline
\caption{Backlog Initial (fin).} \label{backlogv1}\\
\endlastfoot
\hline
\caption[]{Backlog Initial.}\\
\endfoot
1 & L'utilisateur écrit du code dans un éditeur & 200 &  &  &  &  &  \\
\hline
2 & L'utilisateur importe du code dans le logiciel & 1100 & Importer du code Stibbons depuis un fichier, vérifier que le code obtenu est bien identique à celui du fichier. & 4h & 1 & ? & 1h \\
\hline
3 & L'utilisateur visualise les rapports d'erreurs du code & 400 &  &  &  &  &  \\
\hline
4 & L'utilisateur visualise l'évolution du modèle & 1400 & Vérifier que l'interprétation d'instructions données fait bien évoluer comme prévu la tortue dans son environnement. & 24h & 1 & Fini & 70h \\
\hline
5 & L'utilisateur modifie la vitesse (pause, pas à pas, parallèle) & 700 &  &  &  &  &  \\
\hline
6 & Le «dieu-tortue» interprète le code de l'utilisateur & 1500 & Lancer l'interprétation pour~: repeat 4 { fd 1 rt 90 } (suivant syntaxe) ainsi que pour du code avec des erreurs~: repat 4 {...} par exemple ou repeat 4 { fd 1 rt } & 12h & 1 & Fini & 32h \\
\hline
7 & L'utilisateur crée une nouvelle tortue (avec un code) & 600 & Lancer l'interprétation pour~: create-turtle {} et observer une nouvelle tortue apparaître dans l'interface graphique. & 7,5h & 2 &  &  \\
\hline
8 & Les tortues s'exécutent en parallèles & 1200 & Lancer l'interprétation pour deux tortues d'un bout de code et observer l'exécution parallèle via des écritures dans le terminal (Je suis tortue 1 et Je
suis tortue 2 par exemple) & 42,5h & 2 & & \\
\hline
9 & Les tortues communiquent directement entre elles & 900 &  &  &  &  &  \\
\hline
10 & Une tortue se déplace dans l'environnement & 1300 & Ecriture d'instructions simples~: repeat 4 { fd 1 rt 90 } (suivant syntaxe) - Renvoi de la position de la tortue après chaque deplacement~: where\_am\_i(); (suivant syntaxe) & 16h & 1 & Fini & 24h \\
\hline
11 & Les tortues communiquent avec les zones de l'environnement & 1000 &  &  &  &  &  \\
\hline
12 & L'utilisateur exporte le code & 500 &  &  &  &  &  \\
\hline
13 & L'utilisateur exporte le modèle & 300 &  &  &  &  &  \\
\hline
14 & L'utilisateur ajoute une entrée & 100 &  &  &  &  &  \\
\hline
15 & L'utilisateur remet à zéro l'environnement & 800 &  &  &  &  &  \\
\hline
16 & L'utilisateur utilise des variables dans le code & 1275 & Ecrire a = 90 fd a et observer la tortue qui avance. & 12h & 2 &  &  \\
\hline
17 & L'utilisateur définit des fonctions personnalisées dans le code & 1250 & Ecrire function f () { fd 90 } f () et observer la tortue qui avance. & 23,5h & 2 & &  \\
\hline
18 & Les tortues communiquent via l'environnement & 950 &  &  &  &  &  \\
\hline
19 & L'utilisateur utilise des conditionnelles & 550 & Ecrire if(false) { fd 90 } et observer que la tortue ne bouge pas. Refaire le même test avec if(true) et observer que la tortue bouge. & 3,5h & 2 &  &  \\
\hline
20 & L'utilisateur utilise des boucles & 575 & Ecrire pd repeat 4 { fd 40 rt 90 } et observer que la tortue dessine un carré. & 7h & 2 &  &  \\
\hline
\end{longtable}}

Le backlog est composé de fonctionnalités, qui ont un indice entre 1000 et 0 pour leur importance du rendu final.

On écrit aussi un «~scénario utilisateur~» pour chaque fonctionnalité. Cela correspond a une courte description d'une fonctionnalité du programme.

A chaque sprint, on décide des fonctionnalités qu'on ajoute, et du nombre d'heures qu'on estime devoir y passer.

Voici pour comparer le backlog lors du début du sprint 4 (cf.~\ref{backlogsp4}):
{\tiny
\begin{longtable}[c]{|c|p{1.3cm}|c|p{3.5cm}|*{3}{c|}p{0.7cm}|}
\hline
\bf id & \bf Scénario utilisateur & \bf Priorité & \bf Tests & \bf Estimation & \bf Sprint & \bf Statut & \bf Temps réel \\
\hline
\endfirsthead
\hline
\bf id & \bf Scénario utilisateur & \bf Priorité & \bf Tests & \bf Estimation & \bf Sprint & \bf Statut & \bf Temps réel \\
\hline
\endhead
\hline
\caption{Backlog Sprint 4 (fin).} \label{backlogsp4}\\
\endlastfoot
\hline
\caption[]{Backlog Sprint 4.}\\
\endfoot
1 & L'utilisateur écrit du code dans un éditeur & 200 & & & & & \\
\hline
2 & L'utilisateur importe du code dans le logiciel & 1100 & Importer du code Stibbons depuis un fichier, vérifier que le code obtenu est bien identique à celui du fichier. & 4 & 1 & Fini & 1 \\
\hline
3 & L'utilisateur visualise les rapports d'erreurs du code & 400 & Exécuter pd 50 et constater une erreur. & 8h & 4 & & \\
\hline
4 & L'utilisateur visualise l'évolution du modèle & 1400 & Vérifier que l'interprétation d'instructions données fait bien évoluer comme prévu la tortue dans son environnement. & 24h & 1 & Fini & 70h \\
\hline
5 & L'utilisateur modifie la vitesse (pause, pas à pas, parallèle) & 700 & Faire varier la vitesse et observer le changement dans l'execution des tortues. & 12h & 4 & & \\
\hline
6 & Le «dieu-tortue» interprète le code de l'utilisateur & 1500 & Lancer l'interprétation pour~: repeat 4 { fd 1 rt 90 } (suivant syntaxe) ainsi que pour du code avec des erreurs~: repat 4 {...} par exemple, ou repeat 4 { fd 1 rt } & 12h & 1 & Fini & 32h \\
\hline
7 & L'utilisateur crée une nouvelle tortue (avec un code) & 600 & Lancer l'interprétation pour~: create-turtle {} et observer une nouvelle tortue apparaître dans l'interface graphique. & 7,5h & 2 & Fini & 4h \\
\hline
8 & Les tortues s'exécutent en parallèles & 1200 & Lancer l'interprétation pour deux tortues d'un bout de code et observer l'exécution parallèle via des écritures dans le terminal (Je suis tortue 1 et Je suis tortue 2 par exemple) & 42,5h & 2 & Fini & 40h \\
\hline
9 & Les tortues communiquent directement entre elles & 900 & Ecrire send(t,''Je suis là'') avec t une tortue qui a pour code~: m = recv() if (m == ''Je suis là'') fd 50 et observer la tortue avancer & 30h & 3 & Fini & 16h  \\
\hline
10 & Une tortue se déplace dans l'environnement & 1300 & Ecriture d'instructions simples~: repeat 4 { fd 1 rt 90 } (suivant syntaxe) - Renvoi de la position de la tortue après chaque deplacement~: where\_am\_i(); (suivant syntaxe) & 16h & 1 & Fini & 24h \\
\hline
11 & Les tortues communiquent avec les zones de l'environnement & 1000 & Ecrire if(zone.color == red) { color = blue } sur une zone de couleur rouge et observer la tortue changer de couleur. & 30h & 3 & Fini & 12h \\
\hline
12 & L'utilisateur exporte le code & 500 & & & & & \\
\hline
13 & L'utilisateur exporte le modèle & 300 & Sauvegarder le modèle et observer les données en sortie (JSON) & 30h & 4 & & \\
\hline
14 & L'utilisateur ajoute une entrée & 100 &  &  &  &  &  \\
\hline
15 & L'utilisateur remet à zéro l'environnement & 800 & Remettre à zéro après une exécution et observer que le monde est vierge. & 20h & 4 & & \\
\hline
16 & L'utilisateur utilise des variables dans le code & 1275 & Ecrire a = 90 fd a et observer la tortue qui avance. & 12h & 2 & Fini & 3h \\
\hline
17 & L'utilisateur définit des fonctions personnalisées dans le code & 1250 & Ecrire function f () { fd 90 } f () et observer la tortue qui avance. & 23,5h & 2 & Fini & 11h \\
\hline
18 & Les tortues communiquent via l'environnement & 950 & Ecrire broadcast(20,''Je suis là'') et dans une autre tortue dans le rayon avec le code m = recv() if (m == ''Je suis là'') fd 50 et observer la tortue avancer. & 10h & 3 & Fini & 2h \\
\hline
19 & L'utilisateur utilise des conditionnelles & 550 & Ecrire if(false) { fd 90 } et observer que la tortue ne bouge pas. Refaire le même test avec if(true) et observer que la tortue bouge. & 3,5h & 2 & Fini & 3,5h \\
\hline
20 & L'utilisateur utilise des boucles & 575 & Ecrire pd repeat 4 { fd 40 rt 90 } et observer que la tortue dessine un carré. & 7h & 2 & Fini & 3,5h \\
\hline
21 & L'utilisateur définit des fonctions avec paramètres & 1225 & Ecrire function f(a) { fd a } f(90) et observer la tortue avancer. & 10h & 3 & Fini & 16h \\
\hline
22 & L'utilisateur instancie des agents avec paramètres & 560 & Ecrire agent wolf (a) { fd a } new wolf (50) et observer la nouvelle tortue avancer. & 10h & 3 & Fini & 16h \\
\hline
23 & L'utilisateur modifie la couleur d'un agent (tortue ou zone) & 450 & Ecrire new agent { color = red } et observer la nouvelle tortue rouge. & 3h & 3 & Fini & 2h \\
\hline
24 & La tortue accède aux données de son parent & & & & \\
\hline
25 & L'utilisateur peut stocker des valeurs dans une table & 540 & On stocke le retour dans t= {1,2} & 10h & 4 & & \\
\hline
\end{longtable}}

A la fin d'un sprint, on estime le temps passer sur chaque tâche, on fait le point sur notre avancée dans le projet.

Ici, on peut voir un ajout de fonctionnalités, le changement de statut des fonctionnalités réalisées, et le temps réel passer à les rendre utilisables.
Globalement, nous étions toujours dans les temps, car nous surestimions certaines tâches, qui s'averait plus facile que prévue.


