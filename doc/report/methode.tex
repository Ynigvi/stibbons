Pour réaliser ce projet, nous avons choisi d'utiliser la méthode agile SCRUM. Nous nous sommes principalement servi des cours de gestion de projet de ce semestre, en particulier des interventions de Sandrine Maton. Utiliser une méthode agile permet d'avoir un rendu fonctionnel à chaque fin de sprint, et donc de s'assurer d'avoir un rendu final. De plus, ces méthodes étant de plus en plus populaires dans le milieu des entreprises, nous souhaitions donc l'expérimenter.

La méthode Scrum est une méthode itérative, dont chaque itération est nommé «~sprint~». Un sprint se décompose en trois parties~: une réunion initiale, où les objectifs du sprint sont définis et les tâches réparties, le sprint en lui-même, durant lequel le développement a lieu, et une réunion pour faire le point sur le sprint écoulé.
\subsubsection{Backlog initial}
Le sprint 0 correspond à la première période du projet, durant laquelle nous avons fait notre analyse de l'existant et des outils. C'est également durant ce sprint que nous avons choisi les différentes fonctionnalités que nous voulions voir dans le projet. Pour chacune d'entre elles, nous écrivions un scénario utilisateur, une courte description de ce que l'utilisateur doit pouvoir faire via une fonctionnalité. 

Une fois ces fonctionnalités définies, nous avons priorisés celles-ci et établit le «~backlog~»~: la feuille de route de notre projet (cf. Table~\ref{bsp1}).

{\tiny
\begin{longtable}[c]{|c|p{2cm}|c|p{4cm}|*{4}{c|}}
\hline
\bf id & \bf Scénario utilisateur & \bf Priorité & \bf Tests & \bf Estimation & \bf Sprint & \bf Statut & \bf Temps réel \\
\hline
\endfirsthead
\hline
\bf id & \bf Scénario utilisateur & \bf Priorité & \bf Tests & \bf Estimation & \bf Sprint & \bf Statut & \bf Temps réel \\
\hline
\endhead
\hline
\caption{Le backlog au début du sprint 2} \label{bsp1}\\
\endlastfoot
\hline
\caption[]{Le backlog au début du sprint 2}\\
\endfoot
1 & L'utilisateur écrit du code dans un éditeur & 200 &  &  &  &  &  \\
\hline
2 & L'utilisateur importe du code dans le logiciel & 1100 & Importer du code Stibbons depuis un fichier, vérifier que le code obtenu est bien identique à celui du fichier. & 4h & 1 & Fini & 1h \\
\hline
3 & L'utilisateur visualise les rapports d'erreurs du code & 400 &  &  &  &  &  \\
\hline
4 & L'utilisateur visualise l'évolution du modèle & 1400 & Vérifier que l'interprétation d'instructions données fait bien évoluer comme prévu la tortue dans son environnement. & 24h & 1 & Fini & 70h \\
\hline
5 & L'utilisateur modifie la vitesse (pause, pas à pas, parallèle) & 700 &  &  &  &  &  \\
\hline
6 & Le «~dieu-tortue~» interprète le code de l'utilisateur & 1500 & Lancer l'interprétation pour~: repeat 4 { fd 1 rt 90 } (suivant syntaxe) ainsi que pour du code avec des erreurs~: repat 4 {...} par exemple ou repeat 4 { fd 1 rt } & 12h & 1 & Fini & 32h \\
\hline
7 & L'utilisateur crée une nouvelle tortue (avec un code) & 600 & Lancer l'interprétation pour~: create-turtle {} et observer une nouvelle tortue apparaître dans l'interface graphique. & 7,5h & 2 &  &  \\
\hline
8 & Les tortues s'exécutent en parallèles & 1200 & Lancer l'interprétation pour deux tortues d'un bout de code et observer l'exécution parallèle via des écritures dans le terminal (Je suis tortue 1 et Je
suis tortue 2 par exemple) & 42,5h & 2 & & \\
\hline
9 & Les tortues communiquent directement entre elles & 900 &  &  &  &  &  \\
\hline
10 & Une tortue se déplace dans l'environnement & 1300 & Ecriture d'instructions simples~: repeat 4 { fd 1 rt 90 } (suivant syntaxe) - Renvoi de la position de la tortue après chaque deplacement~: where\_am\_i(); (suivant syntaxe) & 16h & 1 & Fini & 24h \\
\hline
11 & Les tortues communiquent avec les zones de l'environnement & 1000 &  &  &  &  &  \\
\hline
12 & L'utilisateur exporte le code & 500 &  &  &  &  &  \\
\hline
13 & L'utilisateur exporte le modèle & 300 &  &  &  &  &  \\
\hline
14 & L'utilisateur ajoute une entrée & 100 &  &  &  &  &  \\
\hline
15 & L'utilisateur remet à zéro l'environnement & 800 &  &  &  &  &  \\
\hline
16 & L'utilisateur utilise des variables dans le code & 1275 & Ecrire a = 90 fd a et observer la tortue qui avance. & 12h & 2 &  &  \\
\hline
17 & L'utilisateur définit des fonctions personnalisées dans le code & 1250 & Ecrire function f () { fd 90 } f () et observer la tortue qui avance. & 23,5h & 2 & &  \\
\hline
18 & Les tortues communiquent via l'environnement & 950 &  &  &  &  &  \\
\hline
19 & L'utilisateur utilise des conditionnelles & 550 & Ecrire if(false) { fd 90 } et observer que la tortue ne bouge pas. Refaire le même test avec if(true) et observer que la tortue bouge. & 3,5h & 2 &  &  \\
\hline
20 & L'utilisateur utilise des boucles & 575 & Ecrire pd repeat 4 { fd 40 rt 90 } et observer que la tortue dessine un carré. & 7h & 2 &  &  \\
\hline
\end{longtable}}

Lors de la réunion de début de sprint, les fonctionnalités à ajouter sont choisies et pour chacune d'entre elles une description du test à effectuer afin de la valider était ajouté au backlog. Nous faisions également une estimation du nombre d'heures nécessaires à chacune des nouveautés.
À la fin d'un sprint, nous estimions le temps passé sur chaque tâche et nous faisions le point sur notre avancée dans le projet.

\subsubsection{Backlog final}

Sur le backlog final (cf. Table~\ref{bsp5}), nous pouvons voir les ajouts effectués, le changement de statut des fonctionnalités réalisées, et le temps réel passer à les rendre utilisables.

{\tiny
\begin{longtable}[c]{|c|p{2cm}|c|p{4cm}|*{4}{c|}}
\hline
\bf id & \bf Scénario utilisateur & \bf Priorité & \bf Tests & \bf Estimation & \bf Sprint & \bf Statut & \bf Temps réel \\
\hline
\endfirsthead
\hline
\bf id & \bf Scénario utilisateur & \bf Priorité & \bf Tests & \bf Estimation & \bf Sprint & \bf Statut & \bf Temps réel \\
\hline
\endhead
\hline
\caption{Le backlog à la fin du sprint 5} \label{bsp5}\\
\endlastfoot
\hline
\caption[]{Le backlog à la fin du sprint 5}\\
\endfoot
1 & L'utilisateur écrit du code dans un éditeur & 200 & L'utilisateur écrit du code dans un éditeur intégré & 20h & 5 & Fini & 19h \\
\hline
2 & L'utilisateur importe du code dans le logiciel & 1100 & Importer du code Stibbons depuis un fichier, vérifier que le code obtenu est bien identique à celui du fichier. & 4h & 1 & Fini & 1h \\
\hline
3 & L'utilisateur visualise les rapports d'erreurs du code & 400 & Exécuter pd 50 et constater une erreur. & 8h & 4 & Fini & 8h \\
\hline
4 & L'utilisateur visualise l'évolution du modèle & 1400 & Vérifier que l'interprétation d'instructions données fait bien évoluer comme prévu la tortue dans son environnement. & 24h & 1 & Fini & 70h \\
\hline
5 & L'utilisateur modifie la vitesse (pause, pas à pas, parallèle) & 700 & Faire varier la vitesse et observer le changement dans l'execution des tortues. & 12h & 4 & Fini & 14h \\
\hline
6 & Le «~dieu-tortue~» interprète le code de l'utilisateur & 1500 & Lancer l'interprétation pour~: repeat 4 { fd 1 rt 90 } (suivant syntaxe) ainsi que pour du code avec des erreurs~: repat 4 {...} par exemple, ou repeat 4 { fd 1 rt } & 12h & 1 & Fini & 32h \\
\hline
7 & L'utilisateur crée une nouvelle tortue (avec un code) & 600 & Lancer l'interprétation pour~: create-turtle {} et observer une nouvelle tortue apparaître dans l'interface graphique. & 7,5h & 2 & Fini & 4h \\
\hline
8 & Les tortues s'exécutent en parallèles & 1200 & Lancer l'interprétation pour deux tortues d'un bout de code et observer l'exécution parallèle via des écritures dans le terminal (Je suis tortue 1 et Je suis tortue 2 par exemple) & 42,5h & 2 & Fini & 40h \\
\hline
9 & Les tortues communiquent directement entre elles & 900 & Ecrire send(t,''Je suis là'') avec t une tortue qui a pour code~: m = recv() if (m == ''Je suis là'') fd 50 et observer la tortue avancer & 30h & 3 & Fini & 16h  \\
\hline
10 & Une tortue se déplace dans l'environnement & 1300 & Ecriture d'instructions simples~: repeat 4 { fd 1 rt 90 } (suivant syntaxe) - Renvoi de la position de la tortue après chaque deplacement~: where\_am\_i(); (suivant syntaxe) & 16h & 1 & Fini & 24h \\
\hline
11 & Les tortues communiquent avec les zones de l'environnement & 1000 & Ecrire if(zone.color == red) { color = blue } sur une zone de couleur rouge et observer la tortue changer de couleur. & 30h & 3 & Fini & 12h \\
\hline
12 & L'utilisateur exporte le code & 500 & L'utilisateur sauvegarde son code dans un fichier externe & 1h & 5 & Fini & 1h \\
\hline
13 & L'utilisateur exporte le modèle & 300 & Sauvegarder le modèle et observer les données en sortie (JSON) & 30h & 4 & Fini & 22h \\
\hline
14 & L'utilisateur ajoute une entrée & 100 &  &  &  &  &  \\
\hline
15 & L'utilisateur remet à zéro l'environnement & 800 & Remettre à zéro après une exécution et observer que le monde est vierge. & 20h & 4 & Fini & 23h \\
\hline
16 & L'utilisateur utilise des variables dans le code & 1275 & Ecrire a = 90 fd a et observer la tortue qui avance. & 12h & 2 & Fini & 3h \\
\hline
17 & L'utilisateur définit des fonctions personnalisées dans le code & 1250 & Ecrire function f () { fd 90 } f () et observer la tortue qui avance. & 23,5h & 2 & Fini & 11h \\
\hline
18 & Les tortues communiquent via l'environnement & 950 & Ecrire broadcast(20,''Je suis là'') et dans une autre tortue dans le rayon avec le code m = recv() if (m == ''Je suis là'') fd 50 et observer la tortue avancer. & 10h & 3 & Fini & 2h \\
\hline
19 & L'utilisateur utilise des conditionnelles & 550 & Ecrire if(false) { fd 90 } et observer que la tortue ne bouge pas. Refaire le même test avec if(true) et observer que la tortue bouge. & 3,5h & 2 & Fini & 3,5h \\
\hline
20 & L'utilisateur utilise des boucles & 575 & Ecrire pd repeat 4 { fd 40 rt 90 } et observer que la tortue dessine un carré. & 7h & 2 & Fini & 3,5h \\
\hline
21 & L'utilisateur définit des fonctions avec paramètres & 1225 & Ecrire function f(a) { fd a } f(90) et observer la tortue avancer. & 10h & 3 & Fini & 16h \\
\hline
22 & L'utilisateur instancie des agents avec paramètres & 560 & Ecrire agent wolf (a) { fd a } new wolf (50) et observer la nouvelle tortue avancer. & 10h & 3 & Fini & 16h \\
\hline
23 & L'utilisateur modifie la couleur d'un agent (tortue ou zone) & 450 & Ecrire new agent { color = red } et observer la nouvelle tortue rouge. & 3h & 3 & Fini & 2h \\
\hline
24 & La tortue accède aux données de son parent & 50 & Taper parent.color= blue et observer le parent devenir bleu & 1h & 5 & Fini & 1h \\
\hline
25 & L'utilisateur peut stocker des valeurs dans une table & 540 & On stocke le retour dans t= \{1,2\} & 10h & 4 & Fini & 9h \\
\hline
26 & L'utilisateur parcours un tableau & 530 & Faire foreach(f~: \{1,2\}) println(f) & 4h & 5 & Fini & 4h \\
\hline
27 & L'utilisateur lance l'application sans interface graphique & 250 & Lancer l'application et regarder les fichiers d'exportations & 16h & 5 & Fini & 15h \\
\hline
\end{longtable}}


