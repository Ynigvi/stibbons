L'analyse lexicale vise à produire un flot de jeton qui pourront être analyser par l'analyseur syntaxique. Ces jetons sont des paires composées d'un type de jeton et de la valeur du lexème (par exemple, l'analyse du lexème \verb|12| va générer le jeton \verb|<NUMBER,12>| dans notre cas). Certains lexèmes peuvent générer des jetons qui n'ont pas de valeur (par exemple, le lexème \verb|(| va entrainer la génération du jeton \verb|<(>|).

Nous avons fait le choix d'utiliser l'outil Flex pour notre projet (cf \ref{Flex}).

\subsection{Jetons}
Notre analyseur lexical a dans un premier temps généré un nombre limité de jetons. Ainsi, en version 0.1, nous générions seulement 26 jetons différents (dont 5 jetons de littéraux), contre 40 jetons en version 1.0 (dont 7 jetons de littéraux).

\subsubsection{}

\subsection{Règles}



