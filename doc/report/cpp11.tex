\section{C++11}
C++11 est la version du standard C++ que nous avons choisi d'utiliser. En effet, cette version a vu apparaître de nombreux ajouts, comme par exemple les \verb|std::thread| et les \verb|std::mutex|, qui nous ont permis de gérer le multithreading sans avoir à nous soucier de leur implémentation (on n'a ainsi pas de problèmes de portabilité liés à ceci).

En outre, d'autres ajouts de C++11, tel que l'inférence de type (avec le mot-clef \verb|auto|) ou les boucles traversant une collection (\verb|for (auto i: collection)|), nous ont permis une écriture plus souple et plus moderne du code.

\section{GDB}
Le «~GNU Project Debugger~» (ref.~\cite{gdb}) est un programme de débogage permettant notamment de tracer les appels de fonctions et de spécifier des conditions d'arrêts lors de l'exécution d'un programme. Il peut donc permettre d'avancer pas à pas dans un programme et de repérer l'endroit où un bogue se situe. Il est très utilisé pour résoudre les erreurs de segmentation grâce à sa précision de traçage~: la ligne exacte de l'erreur est fournie.
