\section{7 avril 2015}
Nous avons fini le second sprint. Nous avons surtout ajouté des choses dans le modèle et dans l'interpréteur :
il s'agissait de la prise en charge du multi-threading et des variables.\\
Coté modèle, on a ajouté les classes Function, Breed, adapté la classe Turtle, ajouté des mutexs dans chaque classe qui le demandait.
Nous avons également rajouté un systéme de "parenté", qui permet à une tortue de connaître son parent et d'avoir accès à ses propriétés.\\
Coté interprète, il a fallu mettre en place le mettre en place les threads, un à chaque "new agent", et les fonctions.
Lors de cette dernière réunion, nous avons discuté de comment afficher les résultats de simulation de notre programme : diagramme, variables tracées avec un journal... Il faudra choisir une simulation de NetLogo, la tester et comparer les résultats des deux applications.\\
Pour le moment, nous devons redémarrer le programme pour lancer un nouveau fichier. Il faudra utiliser yywrap, qui chaîne les fichiers.\\
Pour la taille du monde, nous pensons à faire une taille par défaut si l'utilisateur n'en choisit pas, et lui donné la possibilité d'en choisir une en début de fichier grâce à une syntaxe différente du stibbons.
