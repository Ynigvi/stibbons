\section{17 mars 2015}

Nous avons effectué une réunion avec notre encadrant suite à la fin de notre premier sprint. Certain points on était soulevé tels que la communication inter-agents et le moyen de l'implémenté mais cela concerne la version 0.3, nous devons donc commencé à y réfléchir sans le réaliser.
\\
Le multi-agents de la version 0.2 sera réalisé à l'aide de threads, avec un thread pour chaque tortue.
\\
Le diagramme UML a également été ré-étudié, nous avons donc remarqué certaines différences avec l'implémentation réalisé au cours de la version 0.1, notamment concernant les classes~: 
\begin{itemize}
\item Value~;
\item Type~;
\item Colored~;
\item Boolean~;
\item Color~;
\item Les classes liées à l'interpréteur (Tree,~Interpreter).
\end{itemize}


Certaines conception ont été revu~:
\begin{itemize} 
\item La classe Shape devrai être relié à une Breed, pas à une Tortue : la Breed définira le comportement d'une tortue~;
\item La classe Breed : renommage en TurtleClass~;
\item Les agrégations et compositions entre les classes sont à rectifier.
\end{itemize}


Pour terminer, deux pistes concernant la gestion de l'arbre syntaxique ont été évoqué~:
\begin{itemize}
\item Réaliser un arbre de dérivation avec chaque frère droit qui est un instruction~;
\item Un arbre abstrait en UML.
\end{itemize}
