\section{JSON Spirit}
JSON Spirit (ref.~\cite{JsonSpirit}) est une librairie C++ qui permet de manipuler des fichiers JSON avec du C++.

Il utilise des structures de données C++ comme les vectors ou les maps, pour stocker les objets JSON. 

C'est un outil nouveau, mais assez puissant. Il n'est pas difficile à prendre en main : une classe \verb|Value| existe, et représente n'importe quels types de données (array, object, etc.) et sert de base pour la construction d'objet comme les \verb|Array| de JSON Spirit (un \verb|vector| de \verb|Value|), ou les \verb|Object| (un \verb|vector| de \verb|std::pair| C++).

Grâce à ces emboîtements, on peut représenter un fichier JSON exactement.
