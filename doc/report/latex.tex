\section{Latex}

\subsection{Généralités}
\LaTeX{} est un langage de rédaction de document qui force à avoir une structure sur la forme et le contenu. Il est notamment utilisé lors d'écritures de documents scientifiques car son écriture de contenus complexes (équations, bibliographie, etc.) se manie facilement.
Contrairement à d'autres logiciels de rédaction tel que LibreOffice, OpenOffice, etc., \LaTeX{} n'est pas de type WYSIWYG (What You See Is What You Get). Il faut donc explicité la mise en page du document, d'où sa catégorie de langage.

\LaTeX{} est un logiciel libre.

\subsection{Beamer}
Beamer est un paquet spécialisé de LaTeX pour la création de présentation, souvent sous forme de diapositive. Plusieurs thèmes existent pour la mise en forme et, comme \LaTeX{}, Beamer n'est pas de type WYSIWYG.

