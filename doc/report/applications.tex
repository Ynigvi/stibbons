\section{Application en ligne de commande}

Une application en ligne de commande a été ajoutée à la version 0.4. Elle permet d'exécuter des programmes Stibbons sans serveur graphique et à pleine vitesse.

\subsection{Utilisation}

Utilisation~: stibbons-cli [options] fichier

Options~:
\begin{description}
	\item[-h, \texttt{-{}-}help] Affiche l'aide.
	\item[-v, \texttt{-{}-}version] Affiche l'information de version.
	\item[-e, \texttt{-{}-}export <secondes>] Exporte le modèle toutes les <secondes> secondes.
	\item[-p, \texttt{-{}-}prefix <prefixe>] Préfixe les fichiers exportés avec <prefixe>.
	\item[\texttt{-{}-}png] Génère une image PNG pour chaque export.
	\item[\texttt{-{}-}no-json] N'exporte pas le modèle dans un fichier JSON.
\end{description}

Arguments~:
\begin{description}
	\item[fichier] Le fichier de programme Stibbons à exécuter.
\end{description}

\subsection{Fonctionnement}

Le programme analyse la ligne de commande à l'aide de la classe QCommandLineParser, puis crée et exécute un objet représentant l'application selon les paramètres passés par l'utilisateur.

Lors de l'exécution de l'application, il est possible d'exporter le modèle à intervale régulier grâce à l'option \texttt{-{}-}export. Par défaut, le modèle est exporté dans un fichier JSON, mais il est également possible d'exporter un rendu du monde en PNG, ou de ne pas exporter.

Afin de pouvoir dessiner le monde dans un fichier image, la classe WorldView, widget de l'application graphique jusque là responsable de dessiner le monde sur lui même, a vu ses fonctionnnalités de dessin d'un monde migrer vers la nouvelle classe WorldPainter, capable de dessiner un monde sur une surface.
Cette nouvelle classe est désormais utilisée dans l'application graphique par WorldView pour se dessiner, et par l'application en ligne de commandes pour dessiner sur une QImage.
