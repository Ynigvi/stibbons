\section{Sujet}
	\subsection{Objectif}
	Le projet Stibbons est un projet visant à créer un interprète d'un dérivé de Logo, un langage de programmation permettant l'animation d'une «~tortue~» (cf \ref{Logo}). Nous avons choisi de développer un langage multi-agents, à l'instar de NetLogo ou StarLogo (cf \ref{NetLogo} et \ref{StarLogo}). Ainsi, l'objectif de ce projet était double~: la réalisation d'un interprète, capable d'analyser du code écrit dans un derivé de Logo, ainsi que la réalisation d'une application graphique capable de représenter l'évolution des agents.
	
	\subsection{Système multi-agents}
	Les systèmes multi-agents sont une approche de l'intelligence artificielle visant à facilité la résolution d'un problème en le découplant en plusieurs sous-problèmes. Ainsi, plutôt que de chercher à développer une intelligence unique complexe capable de résoudre le problème, l'approche multi-agent vise plutôt à créer des multitudes d'intelligences capables de résoudre qu'une petite partie du problème, et de compter sur l'intelligence collective émergeante pour voir apparaitre la solution au problème \cite{sma}.

	Ainsi, on peut prendre en exemple les fourmis qui, bien que n'ayant individuellement qu'une capacité «~limitée~», sont capables de survivre grâce à la synergie de leurs colonies.

\section{Cahier des charges}
	\subsection{Fonctionnalités attendues}
	Bien que la méthode de développement utilisée est fait apparaitre de nombreuses fonctionnalités souhaitables au fur à mesure du projet, un certain nombre d'entre elles nous ont semblés indispensable dès le début~:
	\begin{itemize}
		\item chaque agent devait évoluer parallèlement aux autres agents~;
		\item chaque agent devait être capable de communiquer avec les autres agents~;
		\item chaque agent devait pouvoir modifier d'une certaine façon le monde~;
		\item des structures des langages modernes de la programmation impératives devaient être présent dans le langage (tels que les conditionnelles, les boucles, les fonctions, etc.)~;
		\item l'utilisateur devait pouvoir voir le monde où évoluent ces agents.
	\end{itemize}
