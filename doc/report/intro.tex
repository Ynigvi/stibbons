\section{Sujet}
	\subsection{Objectif}
	Le projet Stibbons vise à créer un interprète d'un dérivé de Logo, un langage de programmation permettant l'animation d'un agent mobile, dit «~tortue~» (cf. \ref{Logo}). Nous avons choisi de développer un langage multi-agents, à l'instar de NetLogo ou StarLogo (cf. \ref{NetLogo} et \ref{StarLogo}), rendant ainsi l'objectif de ce projet double~: d'une part la réalisation d'un interprète capable d'analyser du code écrit dans un derivé de Logo, d'autre part la réalisation d'une application graphique capable de représenter l'évolution des agents.

	\subsection{Systèmes multi-agents}
	Les systèmes multi-agents sont une approche de l'intelligence artificielle visant à faciliter la résolution d'un problème en le découplant en plusieurs sous-problèmes. Ainsi, plutôt que de chercher à développer une intelligence unique complexe capable de résoudre le problème, l'approche multi-agents vise plutôt à créer des multitudes d'intelligences capables de résoudre une petite partie du problème, et de compter sur l'intelligence collective émergeante pour voir apparaître la solution au problème \cite{sma}.

	Ainsi, on peut prendre en exemple les fourmis qui, bien que n'ayant individuellement qu'une capacité limitée, sont capables de survivre grâce à la synergie de leurs colonies.

\section{Cahier des charges}
	\subsection{Fonctionnalités attendues}
	Bien que la méthode de développement utilisée ait fait apparaître de nombreuses fonctionnalités souhaitables au fur et à mesure du projet, un certain nombre d'entre elles nous ont semblé indispensables dès le début~:
	\begin{itemize}
		\item chaque agent devait être capable de communiquer avec les autres agents~;
		\item chaque agent devait pouvoir modifier d'une certaine façon le monde~;
		\item le langage devait comprendre des structures des langages de programmation impérative modernes, telles que des conditionnelles, des boucles, des fonctions, etc.~;
		\item l'utilisateur devait pouvoir voir le monde où évoluent ces agents.
	\end{itemize}

        \subsection{Contraintes}
        Nous avions dès le début isolé un certain nombre de contraintes que nous souhaitions pour ce projet~:
        \begin{itemize}
          \item l'utilisation d'un langage non interprété, de type C ou C++, pour son développement~;
		  \item chaque agent devait évoluer parallèlement aux autres agents~;
          \item l'utilisation d'une méthode agile pour développer le projet.
        \end{itemize}
