Durant quatre mois, nous avons réalisé ce projet que nous avions imaginé.
Ce projet a été positif à bien des égards. En effet, il nous a tout d'abord permis de pratiquer une méthode agile, ce qui était une bonne expérience et, avec le recul, un bon choix, car nous avions un projet fonctionnel toutes les deux semaines.
Nous avons pu en outre grâce à cette méthode ajouter facilement de nouvelles fonctionnalités en cours de route, avec notamment l'application en ligne de commande, complémentaire à notre programme principal.

Ce fut également une expérience profitable car nous avons pu voir l'importance de la communication sur ce type de développement. En effet, les cycles étant très courts, il était important de communiquer rapidement et clairement pour éviter de prendre du retard. Si nous avons parfois dépassé légèrement le délai (quelques heures de retard sur le premier sprint), nous avons tout de même sû gérer notre temps, ce qui nous a permis d'être dans les temps sur l'ensemble du projet.

Nous avons également pu grâce à ce projet travailler avec des outils nouveaux (Qt, JsonSpirit, \verb|std::thread|) et nous améliorer sur d'autres outils (Flex, Bison, Pointeurs intelligents).

Bien que la 1.0 soit une version fonctionnelle, il y a cependant des choses à améliorer. La première, et sans doute la plus importante, est le fait que notre parcours d'arbre se fait de façon récursive. Par conséquent, les appels récursifs en Stibbons vont impacter directement la pile d'exécution de l'interprète, et nous limite beaucoup.
De plus, les tortues évoluant chacune dans un thread séparé, on est relativement vite limité par le nombre de tortues pouvant évoluer dans notre programme (changement de contexte lourd).
Il y a également bon nombre de fonctionnalités que nous pourrions ajouter, comme les entrées dans l'interface, un peu à la manière de NetLogo, qui permettraient de modifier des propriétés de façon interactive avec l'exécution, ou encore l'affichage de la sortie standard dans un cadre du programme, voire dans des petites bulles au dessus des tortues.

Nous retiendrons néanmoins de notre projet un bilan positif. En effet, nous avons créé un langage, le Stibbons, ainsi que son interprète et une application graphique permettant de suivre l'évolution du modèle.
