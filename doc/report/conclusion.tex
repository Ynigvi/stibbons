Durant quatre mois, nous avons réalisé ce projet que nous avions imaginé.
Tout d'abord, ce projet nous a permis de pratiquer la méthode Agile, ce qui était une bonne expérience et, avec le recul, un bon choix, car nous avions un projet fonctionnel toutes les deux semaines.

Nous avons su gérer le coté compilation et le coté programmation C++ de ce TER, avec l'aide de notre tuteur.
Nous avons imaginé au milieu du projet l'application sans interface, complémentaire à notre programme principal.
Nous avons découvert de nouveaux outils tel que Qt ou appronfondi des connaissances, comme en C++, ou en flex.
 C'est avec plaisir que nous avons fini la version 1.0 pour la soutenance.

Il y a cependant des choses à améliorer, comme notre abscence de pile dans le programme. Cela limite la récursion du code.
L'écriture de la sortie standard pourrait être afficher dans le programme. Nous voulions ajouter des entrées sur le programme, comme un slider pour faire varier le nombre de tortue. Et éventuellement, l'ajout de bulle pour afficher les messages reçu par chaque tortue serait un plus.

Nous retiendrons néanmoins que nous avons réalisé notre objectif : avoir un interpréteur de notre langage, le Stibbons et une interface visuel du code qui s'exécute.
