\section{Propriétés standard}

Les agents possèdent certaines propriétés par défaut. Ces propriétés varient selon leur type, et peuvent être de simples valeurs en lecture seule, des fonctions en lecture seule, ou encore des propriétés spéciales ayant une sémantique particulière et requérant un type précis.

\subsection{Attributs communs à tous les agents}

\begin{description}
	\item[black] $\rightarrow$ \#000000

	Le noir.

	\emph{Attribut en lecture seule.}

	\item[white] $\rightarrow$ \#ffffff

	Le blanc.

	\emph{Attribut en lecture seule.}

	\item[grey] $\rightarrow$ \#7f7f7f

	Un gris moyen.

	\emph{Attribut en lecture seule.}

	\item[red] $\rightarrow$ \#ff0000

	Le rouge.

	\emph{Attribut en lecture seule.}

	\item[green] $\rightarrow$ \#00ff00

	Le vert.

	\emph{Attribut en lecture seule.}

	\item[blue] $\rightarrow$ \#0000ff

	Le bleu.

	\emph{Attribut en lecture seule.}

	\item[yellow] $\rightarrow$ \#ffff00

	Le jaune.

	\emph{Attribut en lecture seule.}

	\item[cyan] $\rightarrow$ \#00ffff

	Le cyan.

	\emph{Attribut en lecture seule.}

	\item[magenta] $\rightarrow$ \#ff00ff

	Le magenta.

	\emph{Attribut en lecture seule.}

\end{description}

\subsection{Fonctions communes à tous les agents}

\begin{description}
	% print
	\item[print] (value) $\rightarrow$ null

	Imprime une valeur sur la sortie standard.

	\begin{description}
		\item[value] La valeur à imprimer
		\item[Retourne] null
	\end{description}

	% println
	\item[println] (value) $\rightarrow$ null

	Imprime une valeur dans une nouvelle ligne sur la sortie standard.

	\begin{description}
		\item[value] La valeur à imprimer
		\item[Retourne] null
	\end{description}

	% rand
	\item[rand] () $\rightarrow$ number

	Retourne un entier positif au hasard.

	\begin{description}
		\item[Retourne] un entier positif au hazard
	\end{description}

	% random
	\item[random] (min, max) $\rightarrow$ number

	Retourne un entier positif au hasard compris entre les bornes indiquées.

	\begin{description}
		\item[min] La valeur borne minimale inclusive
		\item[max] La valeur borne maximale exclusive
		\item[Retourne] un entier positif au hazard compris entre les bornes indiquées
	\end{description}

	% type_of
	\item[type\_of] (value) $\rightarrow$ type

	Retourne le type d'une valeur.

	\begin{description}
		\item[value] La valeur dont on veut obtenir le type
		\item[Retourne] le type de la valeur
	\end{description}

	% type_of
	\item[size] (table) $\rightarrow$ number

	Retourne le nombre d'éléments contenus dans une table.

	\begin{description}
		\item[table] La table dont on veut connaître le nombre d'éléments
		\item[Retourne] le nombre d'éléments contenus dans la table
	\end{description}
\end{description}

\subsection{Attributs du monde}

\begin{description}
	\item[max\_x] $\rightarrow$ number

	L'abscisse maximale du monde.

	\emph{Attribut en lecture seule.}

	\item[max\_y] $\rightarrow$ number

	L'ordonnée maximale du monde.

	\emph{Attribut en lecture seule.}
\end{description}

\subsection{Fonctions du monde}

\begin{description}
	% ask_zones
	\item[ask\_zones] (function) $\rightarrow$ null

	Exécute une fonction sur chaque zone.

	\begin{description}
		\item[function] La fonction à exécuter sur chaque zone
		\item[Retourne] null
	\end{description}

\end{description}

\subsection{Attributs des tortues}

\begin{description}
	\item[color] $\rightarrow$ \#000000

	La couleur de la tortue.

	\emph{Requiert une valeur de type color.}

	\item[parent] $\rightarrow$ agent

	L'agent qui a créé la tortue.

	\emph{Attribut en lecture seule.}

	\item[pos\_x] $\rightarrow$ number

	L'abscisse de la position de la tortue.

	\emph{Requiert une valeur de type number.}

	\item[pos\_y] $\rightarrow$ number

	L'ordonnée de la position de la tortue.

	\emph{Requiert une valeur de type number.}

	\item[pos\_angle] $\rightarrow$ number

	L'angle de la tortue.

	\emph{Requiert une valeur de type number.}

	\item[world] $\rightarrow$ world

	Le monde.

	\emph{Attribut en lecture seule.}

	\item[zone] $\rightarrow$ zone

	La zone dans laquelle est la tortue.

	\emph{Attribut en lecture seule.}
\end{description}

\subsection{Fonctions des tortues}

\begin{description}
	% distance_to
	\item[distance\_to] (turtle) $\rightarrow$ number

	Retourne la distance la plus courte vers une autre tortue.

	\begin{description}
		\item[turtle] La tortue vers laquelle obtenir la distance
		\item[Retourne] null
	\end{description}

	% face
	\item[face] (turtle) $\rightarrow$ null

	Tourne la tortue pour qu'elle fasse face à une autre par le chemin le plus court.

	\begin{description}
		\item[turtle] La tortue à laquelle faire face
		\item[Retourne] null
	\end{description}

	% in_radius
	\item[in\_radius] (distance) $\rightarrow$ table

	Retourne l' ensemble de tortues dans le rayon donné autour de la tortue.

	\begin{description}
		\item[distance] Le rayon autour de la tortue à sonder
		\item[Retourne] Une table contenant les tortues dans le rayon
	\end{description}

	% inbox
	\item[inbox] () $\rightarrow$ number

	Retourne le nombre de message non lus.

	\begin{description}
		\item[Retourne] Le nombre de messages non lus
	\end{description}

	% teleport
	\item[teleport] (x, y, angle) $\rightarrow$ null

	Téléporte une tortue à une certaine coordonnée et avec un certain angle.

	\begin{description}
		\item[x] L'abscisse où téléporter la tortue
		\item[y] L'ordonnée où téléporter la tortue
		\item[angle] L'angle à donner à la tortue
		\item[Retourne] null
	\end{description}
\end{description}

\subsection{Attributs des zones}

\begin{description}
	\item[color] $\rightarrow$ \#ffffff

	La couleur de la zone.

	\emph{Requiert une valeur de type color.}

	\item[parent] $\rightarrow$ agent

	L'agent qui a créé la zone.

	\emph{Attribut en lecture seule.}

	\item[world] $\rightarrow$ world

	Le monde.

	\emph{Attribut en lecture seule.}
\end{description}

